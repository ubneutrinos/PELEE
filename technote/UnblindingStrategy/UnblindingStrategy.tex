\newpage
\section{Unblinding strategy}
(Giuseppe, work in progress)

\begin{itemize}
    \item Validate the selection variables in the new run periods (see Sec.~\ref{sec:selvalid}).
    \item Verify the selection makes sense on the small open datasets.
    \item Study sidebands (2-shower selection, events failing the BDT cut, events with higher-energy electrons) along the lines of studies performed for the first result (see Sec.~\ref{sec:sidebands}).
    \item Do we want to consider updating the constraint procedure in case of specific tension in sidebands? For instance, if the 2-shower sideband shows a trend vs the pi0 energy, include it in the constraint? Well, we already know it, but with more data the effect may become more significant.
    \item Once results in the sidebands are understood, in case of no pending issues and when all statistical procedures are settled (see Sec.~\ref{sec:sensitivity}), we'll proceed to unblinding the signal region. 
    \item List here the plots that will be part of the "result" and the additional validation checks we'll do post-unblinding.
    \item Consider discuss a "flow chart" of the ublinding process, and what actions are taken depending on the outcome of the analysis.
\end{itemize}