\section{Systematics and expected sensitivity}
(outline by David)

\subsection{Systematics: unchanged}What is the same as Run 1-2-3 analysis: few sentences + reference past note.
\subsection{Systematics: What's new}
\subsubsection{GENIE FSI bug-fix} A few validation plots, e.g. Chris' validation work of GENIE weights.
\subsubsection{EXT smoothing} (Alex, in progress)
At the final level of the event selection, the number of EXT events is much too small to reliably produce a non-zero bin count in the analysis histograms. In the previous round of the analysis, an error of 1.4 events was assigned to any bin without any EXT events. This number was derived as the expected error given a uniform prior distribution between 0 and 1 on the expectation value. However, this error assignment could lead to an overly conservative error estimate. For this round, we instead use a KDE to produce a smoothed histogram and calculate the covariance matrix of the smoothed histogram using a bootstrapping technique.
\subsubsection{DetVar systematics} Are we assuming Run123 detvars for Run 1-5? Probably true for first sensitivites. \\
Include a table which lists MC stats available on DetVar samples.
\subsubsection{Signal model systematics}

\subsection{systematics summary}

The total systematics error budget is shown in the table of Fig.~\ref{fig:systematicsbudget}.

\begin{center}
\begin{figure}[h]
    \includegraphics[width=1.00\textwidth]{technote/SystematicsSensitivity/Figures/systematicsbudget.png}
    \caption{Enu spectrum before/after constraint.}
    \label{fig:systematicsbudget}
\end{figure}
\end{center}

\newpage
\subsection{Constraints}
Describe briefly, refer to older analysis.

Show spectrum of nues before and after constraint. See Fig.~\ref{fig:constraint}.

\begin{center}
\begin{figure}[h]
    \includegraphics[width=1.00\textwidth]{technote/SystematicsSensitivity/Figures/constraint.png}
    \caption{Neutrino energy spectrum before/after constraint.}
    \label{fig:constraint}
\end{figure}
\end{center}

Show quantitative predicted events and uncertainty (including reduction in uncertainty) before/after constraint. See Table of Fig.~\ref{fig:constrainttable}.

\begin{center}
\begin{figure}
    \includegraphics[width=1.00\textwidth]{technote/SystematicsSensitivity/Figures/constrainttable.png}
    \caption{Impact of constraint.}
    \label{fig:constraint} 
\end{figure}
\end{center}

Potentially think of new ideas here

\newpage
\subsection{Sensitivity}
\label{sec:sensitivity}

\subsubsection{Simple Hypothesis Test}

Distribution of test statistic for backgronud-only and signal models, with extracted median sensitivity. See. Fig.~\ref{fig:simplehypothesis}.

\begin{center}
\begin{figure}[h]
    \includegraphics[width=1.00\textwidth]{technote/SystematicsSensitivity/Figures/simplehypothesis.png}
    \caption{Impact of constraint.}
    \label{fig:simplehypothesis} 
\end{figure}
\end{center}

Fig.~\ref{fig:simplehypothesisresults} shows the expected sensitivity for the simple hypothesis test.

\begin{center}
\begin{figure}[h]
    \includegraphics[width=1.00\textwidth]{technote/SystematicsSensitivity/Figures/simplehypothesisresults.png}
    \caption{Impact of constraint.}
    \label{fig:simplehypothesisresults} 
\end{figure}
\end{center}

\newpage
\subsubsection{Signal strength fit}

The signal strength sensitivity results are shown in Fig.~\ref{fig:signalstrengthsensitivity}.
\begin{center}
\begin{figure}[h]
    \includegraphics[width=1.00\textwidth]{technote/SystematicsSensitivity/Figures/signalstrengthsensitivity.png}
    \caption{Impact of constraint.}
    \label{fig:signalstrengthsensitivity} 
\end{figure}
\end{center}

\newpage
\subsubsection{Validation of sensitivity vs. SBNFit benchmark}
