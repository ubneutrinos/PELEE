\section{Introduction}
\label{sec:Introduction}

\textcolor{blue}{MicroBooNE's first round of results searching for an excess of electron neutrinos consistent with the MiniBooNE ``Low Energy Excess'' (LEE) showed consistency with no excess. The PeLEE analysis in particular rejected the $\nu_e$ electron excess interpretation in the 1$e$N$p$ channel at 97\% CL while observing a mild excess (compatible within uncertainty) in the 1$e$0$p$ channel in bins dominated by $\pi^0$ backgrounds. Yet, many open questions about the MiniBooNE LEE remain, including in the $\nu_e$ channel. These include:
\begin{itemize}
    \item Resolution for the mild discrepancy between data and simulation observed in the electron final-state with no hadronic activity (1$e$0$p$0$\pi$).
    \item An improved understanding of the overall data-simulation agreement at intermediate neutrino energies of $\sim0.7-1.0$ GeV.
    \item Stronger overall statistical conclusions for the PeLEE pionless channels with a sensitivity goal of $\geq 3\sigma$.
    \item An expanded scope of the $\nu_e$ LEE search which includes interpretation of a $\nu_e$ excess consistent with the kinematics of excess events observed in MiniBooNE as well as a scrutiny of possible data-simulation discrepancies in the context of neutrino cross-section modeling.
\end{itemize}
These items motivate performing an updated analysis which incorporates MicroBooNE's full-dataset statistics in addition to targeted analysis developments.
}

The PELEE Statistical Update\todo{suggestion: ``PeLEE Statistical Update'' to ``PeLEE full-dataset analysis''} is an upgraded version of the PELEE analysis where we add \textcolor{blue}{data from} runs 4 and 5. In addition to \sout{the addition of} more statistics, some improvements to the event selection, signal modelling and treatment of systematics have also been made that are documented here. 

\subsection{Goals of this analysis}
\label{sec:Goals}

Goal of this analysis is to perform a statistical update of the first pionless eLEE analysis, thus checking with a larger dataset the stability of the results and comparing the data to a signal model that relies less on assumptions about the physics process leading to the excess in MiniBooNE.

Therefore, updates to the analysis are primarily of statistical nature, either in terms of the amount of data included, or in terms of the statistical procedures used to extract the result:
\begin{itemize}
    \item Include the full MicroBooNE BNB data, from run1 to run5.
    \item Update the signal model of the MiniBooNE excess, now including a model based on the electron kinematics.
    \item Update the statistical treatment of the cosmic background (estimated with triggered beam-off, "EXT", events).
    \item Update the constraint selection, separating the 0p and Np selections and including a selection to constrain the $\pi^0$ background.
    \item Updates to other statistical procedures, such as correlations for detector variations.
\end{itemize}
The only update to the nue selection we consider in this update is the usage of the CRT veto for run3-5. This selection was already used in the numu selection (for constraint) in the first round of the analysis and it makes sense to use it also for the nue selection given that all the additional data include the CRT detector.

While the bulk of the work for this analysis is complete, a few developments need to be fully integrated in the chain. These are listed in Sec.~\ref{sec:todo}.

Further updates to the reconstruction and to the selection will be included in the final version of the analysis, which is planned for 2025.

\subsection{Data and MC Simulation Samples}

The system of MC simulation and data samples used is similar to that of the previous PeLEE and other MicroBooNE analyses: samples of on-beam data and compared to the sum of predictions made using the following:
%
\begin{itemize}
    \item A standard overlay sample (with CC-$\nu_e$ interactions removed), containing neutrino interactions inside the cryostat.
    \item A ``Dirt'' sample, containing neutrino interactions outside of the cryostat.
    \item An ``EXT'' sample, created using data collected when the beam was inactive and used to simulate the background due to cosmic rays mistakenly reconstruted as neutrino interactions.
    \item An ``Intrinsic $\nu_e$'' sample, containing exclusively CC-$\nu_e$ interactions inside the cryostat, employed to provide enough events to properly study the behaviour of this channel.
\end{itemize}
%
The predicted signal from the LEE is generated by reusing events from the intrinsic $\nu_e$ sample, with weights applied to obtain the correct shape of the signal. 

The data collected during each data taking is compared with a separate set of MC simulation samples, including EXT data collected during the same running period, and overlay prepared by combining MC simulation with off-beam data from that period. \textbf{In the current version of this analysis, we do not use data from run 4a}.

\textcolor{blue}{To better understand how the additional statistics, and the availability of the CRT for Runs 4-5 impact the analysis updates, table~\ref{tab:POT} summarizes what POT is available for different Run periods, and how it is used in the analysis.}

\begin{center}
\begin{tabular}{||c | c | c | c | c||} 
 \hline
 Run period & POT & POT w/ CRT & available for analysis? & used in PeLEE \\ [0.5ex] 
 \hline\hline
 Run 1 ``open trigger'' & XXX & XXX & no & no  \\ 
 \hline
  Run 1 & XXX & XXX & yes & yes \\ 
 \hline
   Run 2 & XXX & XXX & yes & yes \\ 
 \hline
   Run 3 & XXX & XXX & yes & yes \\ 
 \hline
   Run 4a & XXX & XXX & yes & no \\ 
 \hline
   Run 4bcd & XXX & XXX & yes & yes \\ 
 \hline
   Run 5 & XXX & XXX & yes & yes \\ 
 \hline\hline
   total data collected & XXX & XXX & & \\ 
 \hline
    total data used & XXX & XXX & & 
 \hline
 \label{tab:POT}
\end{tabular}
\end{center}
